\chapter{Podsumowanie}
    {\em \quad Proces planowania, budowania oraz testowania aplikacji PhotoLab dobiegł końca. W rozdziale tym zostanie podsumowana wykonana praca oraz nastąpi porównanie produkt końcowego z jego wstępnymi założeniami. Dodatkowo omówione będą możliwości dalszego rozwoju i rozszerzania funkcjonalności aplikacji. Na sam koniec rozpisane zostaną wnioski, które nasunęły się w trakcie trwania projektu jak i podczas użytkowania i testowania produktu finalnego.}
    
\section{Efekt końcowy}
\quad Ostatecznie pomimo wielu problemów i przeciwności udało się zakończyć proces budowania aplikacji. Efektem prac jest aplikacja internetowa dostępna dla każdego zainteresowanego użytkownika z poziomu przeglądarki internetowej - zarówno z komputerów klasy PC jak i~urządzeń mobilnych. \\
\\
Zanim zostanie omówione to, co udało się zrobić w ramach projektu, warto przyjrzeć się napotkanym błędom i trudnościom.\\
Pierwszym problemem, który pojawił się już w początkowej fazie procesu implementacji było zbudowanie odpowiedniego środowiska, w którym mógłby być budowany projekt. Wariantów było kilka, trzy z nich zostały przetestowane i spośród nich wybrano ostatecznego zwycięzcę. Pierwszym wyborem było gotowe środowisko wraz z przykładową aplikacją oferującą wprost ,,z pudełka'' bardzo wiele możliwości. Niestety jak się okazało, zbyt wiele, a w dodatku takich, które nie zostałyby wdrożone w aplikacji \textit{PhotoLab}. Projekt ten choć z założenia służył jako ,,aplikacja startowa'' dla nowych projektów, okazał się zbyt przeładowany niepotrzebnymi funkcjami oraz napisany w niespójny i nieuporządkowany sposób, a w dodatku bez dokumentacji projektowej, przez co był bardzo trudny do zrozumienia i dalszej pracy nad nim. \\
Kolejny wybór padł na podstawowy szablon generowany z poziomu tworzenia nowego projektu aplikacji \textit{Visual Studio} - \textit{AngularSPA}. Projekt ten automatycznie generował podwaliny pod aplikację opartą o \textit{Angular'a} po stronie klienta oraz \textit{ASP.NET Core} po stronie serwera. Problemem jednak okazał się mało przejrzysty schemat łączenia obu aplikacji, które mocno zazębiały się swoją strukturą i odpowiedzialności. Dodatkowo \textit{Angular} oparty był na generowaniu po stronie serwera (\textit{ang. Server Side Rendering (SSR)}), nie natomiast po stronie przeglądarki internetowej (utrudniając przez to dostęp do takich funkcjonalności jak np. pamięć lokalna przeglądarki), jak było to zakładane i wymagane dla wdrażanej aplikacji. Dodatkowym problem okazało się narzędzie \textit{WebPack} służące do zarządzania zależnościami między modułami i budowania aplikacji. Nie było ono proste, ani intuicyjne dla programisty, przez co okazało się kolejnym powodem do wycofania się z implementacji tej struktury projektowej. \\
Trzeci wybór, który okazał się wyborem ostatecznym, padł na osobne wygenerowanie aplikacji serwerowej \textit{ASP.NET Core Web API} z poziomu interfejsu środowiska programistycznego \textit{Visual Studio 2017} oraz aplikacji klienckiej. Ta druga powstała w oparciu o bardzo proste narzędzie konsolowe \textit{Anglular CLI}, które nie tylko pomogło wygenerować projekt warstwy prezentacji ale także odpowiednio nim zarządzać. Struktura projektowa utworzona za pomocą tego narzędzia choć do zarządzania zależnościami również korzystała z narzędzia \textit{Webpack}, to było ono obudowane dodatkowym uproszczonym interfejsem \textit{CLI}, który w zupełności wystarczył na potrzeby projektu, natomiast znacznie uprościł proces przygotowywania środowiska.\\
\\
Już na etapie końca implementacji zasadniczej części warstwy serwerowej została podjęta decyzja o aktualizacji \textit{ASP.NET Core} do najnowszej wersji, która w tamtym czasie się pojawiła - wersji \textit{2.0} (z \textit{v1.1}). Miała to być drobna zmiana, nie mająca większego wpływu na obecnie stworzony kod. Rzeczywistość okazała się jednak inna. W implementacji wielu metod, struktur i bibliotek zaszły zmiany (często bardzo drobne), które jednak powodowały, że już wcześniej napisany kod przestał działać. Dość sporą ilość czasu zajęło przywrócenie wcześniej już zaimplementowanych funkcjonalności. Dodatkowo późniejsze budowanie kolejnych elementów serwera okazało się być trudniejsze, niż się spodziewano. W związku z~pracą w~dopiero co wdrożonej wersji technologii, wiele problemów dla których rozwiązanie można bardzo szybko znaleźć w Internecie, okazało się problemami, które należy rozwiązać samemu. Było to spowodowane brakiem stosownej dokumentacji, poradników i ogólnego \textit{community}.\\
\\
Ostatni z poważnych problemów został napotkany na etapie wdrożenia projektu na serwer produkcyjny. Napotkano wówczas na wiele trudności, niezgodności i braków kompatybilności. Ponieważ środki dostępne na wdrożenie były mocno ograniczone, zakres możliwości dotyczący wyboru konkretnych ofert hostingowych również był bardzo wąski. Dodatkowo połączenie aplikacji opartej o dwie różne technologie - \textit{.Net} i \textit{Javascript} nie ułatwiał tego zadania. Pierwsze próby zakładały upublicznienie aplikacji \textit{PhotoLab} jako jednej, scalonej aplikacji. Niestety w~tym wariancie wystąpiło wiele problemów związanych z przekierowywaniem połączeń, wysyłaniem zapytań \textit{HTTP} oraz ogólnym \textit{routingem} aplikacji. Wówczas została podjęta o~osobnym wystawieniu aplikacji klienckiej oraz serwerowej z bazą danych. Po rozwiązaniu pomniejszych problemów rozwiązanie to okazało się poprawne, a wdrożenie zakończyło się pełnym sukcesem.\\
\\
Określając ostateczny efekt projektu należy powiedzieć, iż została zbudowana średnich rozmiarów, \textit{responsywna} aplikacja webowa, której struktura oparta jest o model trójwarstwowy. Interfejs klienta jest przyjazny i łatwy w użytkowaniu, co zostało przetestowane i wykazane w~trakcie przeprowadzania testów użyteczności. Część serwerowa jest spójna i zabezpieczona. Testy wydajnościowe wykazały lukę w jej budowie, która została poprawiona, a ponownie przeprowadzone badania potwierdziły jej odporność na błędy i optymalność działania przy maksymalnym zakładanym obciążeniu.

\section{Możliwości dalszego rozwoju aplikacji}
\quad Aplikacja \textit{PhotoLab} nie jest produktem ostatecznym. Można oznaczyć ją jako wersję \textit{1.0} z~możliwością dalszego rozwoju. Zarówno w trakcie implementacji, jak i w czasie testowania produktu końcowego zauważono bardzo wiele możliwości rozbudowy, modyfikacji lub poprawienia niektórych rozwiązań, a także dodania nowych funkcjonalności. \\
\\
Listę pomysłów i możliwości na dalszy rozwój aplikacji warto rozpocząć od zidentyfikowanych części aplikacji, które wymagają udoskonalenia lub rozbudowy. Z pewnością należą do nich:
\begin{itemize}
    \item Operacje na przesyłanych zdjęciach - obecnie jedyną możliwością edycji zdjęcia przed ich przesłaniem do laboratorium jest ustalenie sposobu dopasowania (białe paski lub wypełnienie) oraz orientacji (pionowej lub poziomej). Moduł ten można by rozszerzyć o takie funkcjonalności jak: przycinanie zdjęć w wizualnym edytorze na żywo, sterowanie kolorystyką zdjęcia (nasycenie, kontrast) oraz gotowe \textit{presety} takie jak: czerń i biel, sepia czy negatyw.
    \item Źródło pochodzenia zdjęć - z pewnością ciekawym dodatkiem byłaby możliwość przesyłania zdjęć do wydruku nie tylko z urządzenia, na którym otwarta jest aplikacja ale również z popularnych portali społecznościowych, czy dysków internetowych (\textit{DropBox}, \textit{Google Drive}, \textit{Mega.nz} etc.).
\end{itemize}

\quad Aplikacja została oparta o architekturę, która umożliwia jej łatwy rozwój, w ramach którego możliwe jest dobudowywanie nowych modułów i funkcjonalności. Kilka propozycji, które nasunęły się podczas testów aplikacji to:
\begin{itemize}
    \item Rejestracja użytkowników - bardzo użytecznym dodatkiem w kolejnej wersji aplikacji byłaby z pewnością możliwość rejestracji jak i logowania użytkowników poprzez popularne portale takie jak \textit{Facebook} czy \textit{Google}. Z pewnością ułatwiło i przyspieszyłoby to proces rejestracji nowych użytkowników.
    \item Dokonywanie płatności - rozszerzenie możliwości np. poprzez system \textit{PayU} lub \textit{PayPal}. Ulepszyłoby to proces opłacania nowych zamówień, a przez to również skróciło czas ich realizacji.
    \item Moduł mailingowy - część aplikacji odpowiedzialna za wszelkiego rodzaju maile rozsyłane w ramach usług serwisu. W celu poinformowania klienta o etapie realizacji zlecenia mógłby być generowany automatyczny mail informacyjny lub administrator systemu z poziomu swojego kokpitu mógłby wysyłać spersonalizowane maile oparte o przygotowane szablony.
\end{itemize}
Jak widać, poprzez zaprezentowane przykłady i pomysły, możliwości kontynuacji prac nad rozwojem kolejnych wersji aplikacji \textit{PhotoLab} jest bardzo wiele. Jedne z nich są bardziej użyteczne, inne mniej, jednak każde wnoszą nowe wartości i możliwości dla funkcjonowania strony. 

\section{Wnioski}
\quad Projekt aplikacja \textit{PhotoLab}, oficjalnie nazwany jako: ,,System do obsługi laboratorium fotograficznego z możliwością sprzedaży foto usług online'', zakończył się sukcesem. Wszystkie zaplanowane funkcjonalności, zarówno podstawowe jak i rozszerzone zostały zaimplementowane i wdrożone w środowisku produkcyjnym. Aplikacja została zbudowana w oparciu o~architekturę, którą starannie zaplanowano już w czasie przygotowań. Strona \textit{PhotoLab} ma dokładnie taką formę i prezencję, jak zostało to uprzednio ustalone. Wszystkie przeprowadzone testy zarówno wydajnościowe, jak i funkcjonalne zakończyły się sukcesem, a testy użyteczności zostały pomyślnie zaakceptowane przez biorących w nich udział użytkowników.\\
\\
Projekt, chociaż zakończony sukcesem, okazał się bardzo trudny i wymagający. Nie obyło się bez wielu problemów i trudności, które zostały dokładnie opisane w \textit{rozdziale 6.1}. Do najważniejszych z nich z pewnością należy zaliczyć aktualizację środowiska serwerowego ASP.NET Core z wersji \textit{1.1} do \textit{2.0}, a także wdrożenie projektu na serwer produkcyjny.\\
\newpage
\noindent Patrząc na cały projekt z perspektywy już gotowego i wdrożonego rozwiązania można stwierdzić, iż bardzo dobrym wyborem było zastosowanie technologii \textit{Angular}. Jest ona niezmiernie szybka i wprowadza  możliwości podejmowania licznych interakcji użytkownika ze stroną poprzez zastosowanie funkcjonalności \textit{two way data binding}. Jednocześnie jest relatywnie prosta we wdrożeniu i implementacji - głównie poprzez zastosowanie narzędzia \textit{Angular CLI}.\\
\\
Analizując stronę serwera aplikacji można z jednej strony zastanawiać się nad słusznością aktualizacji wersji w trakcie implementacji projektu, z drugiej - być bardzo zadowolonym z wyboru jednej rodziny rozwiązań do implementacji zarówno serwera, jak i warstwy bazy danych. Z pewnością umożliwiło to uniknięcia problemów z brakiem odpowiedniego współdziałania i wymiany informacji. Zastosowanie narzędzia klasy \textit{ORM}, \textit{Entity Framework}, znacznie ułatwiło i przyspieszyło czas implementacji, dodatkowo udostępniając bardzo łatwy, prosty i intuicyjny sposób operowania na modelach zarówno dla strony serwera. jak i bazy danych, unikając w ten sposób konieczności posługiwania się kolejnym językiem programowania jakim jest \textit{SQL}.\\
\\
Podsumowując projekt można stwierdzić, iż wszystkie założone funkcjonalności, zarówno podstawowe, jak i rozszerzone zostały zaimplementowane i wdrożone. \textit{PhotoLab} jest w pełni funkcjonalnym systemem zlecania zamówień wydruku zdjęć online dającym użytkownikowi możliwość w łatwy i przyjazny dla niego sposób zapoznać się z firmą oraz jej obszarem działalności, a także skorzystać z oferowanych przez nią usług bez wychodzenia z domu.